\documentclass{article}
    % General document formatting
    \usepackage[margin=0.7in]{geometry}
    \usepackage[parfill]{parskip}
    \usepackage[utf8]{inputenc}
    \usepackage{graphicx}
    \usepackage{fancyhdr}

    % Related to math
    \usepackage{amsmath,amssymb,amsfonts,amsthm}

    % Page Style
    \pagestyle{fancy}
    \fancyhf{}
    \rhead{Johannes Stricker}
    \lhead{Artificial Intelligence: Create A Domain-Independent Planner - Research Review}
    \cfoot{\thepage}

\begin{document}

With this essay a brief overview on the history of AI planning and search between the
1970's and early 2000's is presented and differences between Partial Order Planning
and State Space Planning are explained.

\section*{The history of AI planning and search}
In 1971 Richard E. Fikes and Nils J. Nilsson released a paper titled 'STRIPS: A New
Approach to the Application of Theorem Proving to Problem Solving', which up until
today layed the foundation for most of the research in the field of AI planning
and search.\\
It was the first time that a problem solving model was described using a language
of operators with pre- and postconditions, that operate on the world model. The
idea behind STRIPS was to search for ``some composition of operators that transforms a
given world model into one that satisfies some stated goal condition.`` \cite{STRIPS}
Instead of exploring the complete space of operator combinations - which would grow
too large too quickly even for simple problems - the idea behind STRIPS is to identify
operators that reduce the difference between the current world state and the goal state.
To identify such operators, STRIPS already used heuristic evaluation functions such
as the number of remaining goals.
When an operator has been found, the subproblem of getting to a world state that
satisfies the operators preconditions has to be solved. This process is then applied
iteratively, until all goals have been satisfied.

\\\\

The work of Fikes and Nilsson built the basis for many domain-independent planning
algorithms that were developed between 1971 and the 90's: HACKER, INTERPLAN, NOAH
and TWEAK, just to name a few. \cite{chapman87}
Except for the addition of constraints, none of the above approaches improved STRIPS in any significant
way, though. \cite{chapman87} They were all based on the idea of Partial Order Planning, which
is to find a plan that satisfies all goal requirements, but does not specify an
exact order of actions where it is not required. Partial Order Planners work by searching
for a solution among the set of partial-order plans. This makes sense, because sub-problems
are often independent of each other.
In 1997 Avrim L. Blum and Merrick L. Furst developed a novel approach to solving planning problems
that was based on a structure called Planning Graph. \cite{blum97} The algorithm first
creates a graph structure from the problem at hand, that can then be used for heuristic evaluations
and to search for a partial-order plan.
While Partial Order Planning algorithms have several benefits - e.g. being easier to parallelize - faster methods
emerged in the late 90's, such as State Space Planners and Constraint Satisfaction Solvers,
so that Partial Order Planning received less attention. \cite{AIMA}

\\\\

After the hype around Partial Order Planning, the interest in State Space Planning rose with the release of Drew McDermott's UnPOP
program in 1996, which was successively improved upon by several other projects. One of them was the Heuristic
Search Planner (HSP) that was developed by Bonet and Geffner in 1999.
The idea was similar to preceding work in the field: to have a general mechanism to extract heuristics and a general program
that is able to solve problems domain-independently while being competitive even with domain-specific solvers. \cite{bonet00}
Contrary to searching a solution among the set of partial-order plans, State Space Planners
search among the set of world states, though. Nonetheless the algorithm presented by Bonet and Geffner, as well
as most others, were still using the initial formulation of the STRIPS encoding.
The authors themselves critized however, that ``few real problems can actually be
encoded efficiently in Strips`` \cite{bonet00} and proposed to make use of richer
problem representations in the future.

\\\\

With the new millenial, Partial Order Planning received some attention again. The RePOP
algorithm incorporated novel heuristics that were also used in State Space Planning
in order to improve upon UCPOP from 1992. The evaluation showed that RePOP performed well
in parallel planning domains, but had unsatisfactory performance in serial domains. \cite{nguyen01}


\begin{thebibliography}{1}
\bibitem[Yikes and Nilsson (1971)]{STRIPS} Richard E. Yikes and Nils J. Nilsson {\em STRIPS: A New Approach
to the Application of Theorem Proving to Problem Solving}  1971.

\bibitem[Chapman (1987)]{chapman87} David Chapman {\em Planning for Conjunctive Goals} 1987.

% \bibitem[Penberthy and Wed (1992)]{UCPOP} J. Scott Penberthy and Daniel S. Weld {\em UCPOP: A Sound, Complete,
% Partial Order Planner for ADL} 1992.

\bibitem[Bonet and Geffner (2000)]{bonet00} Blai Bonet and Hector Geffner {\em Planning as heuristic search} 2000.

\bibitem[Nguyen and Kambhampati (2001)]{nguyen01} XuanLong Nguyen and Subbarao Kambhampati {\em Reviving Partial
Order Planning} 2001.

\bibitem[Russel and Norvig (2010)]{AIMA} Stuart Russel and Peter Norvig {\em Artificial Intelligence - A Modern
Approach 3rd Edition} 2010.

\bibitem[Blum and Furst (1997)]{blum97} Avrim L. Blum and Merrick L. Furst {\em Fast Planning Through Planning Graph
Analysis} 1997.
\end{thebibliography}
\end{document}
