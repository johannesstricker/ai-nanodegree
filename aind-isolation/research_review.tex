%----------------------------------------------------------------------------------------
%	PACKAGES AND OTHER DOCUMENT CONFIGURATIONS
%----------------------------------------------------------------------------------------

\documentclass{article}

\usepackage{fancyhdr} % Required for custom headers
\usepackage{lastpage} % Required to determine the last page for the footer
\usepackage{extramarks} % Required for headers and footers

% Margins
\topmargin=-0.45in
\evensidemargin=0in
\oddsidemargin=0in
\textwidth=6.5in
\textheight=9.0in
\headsep=0.25in

\linespread{1.1} % Line spacing

% Set up the header and footer
\pagestyle{fancy}
\lhead{\authorName} % Top left header
\chead{\pageTitle} % Top center header
\rhead{\courseName} % Top right header
\lfoot{\lastxmark} % Bottom left footer
\cfoot{} % Bottom center footer
\rfoot{Page\ \thepage\ of\ \pageref{LastPage}} % Bottom right footer
\renewcommand\headrulewidth{0.4pt} % Size of the header rule
\renewcommand\footrulewidth{0.4pt} % Size of the footer rule

%----------------------------------------------------------------------------------------
%	NAME AND CLASS SECTION
%----------------------------------------------------------------------------------------

\newcommand{\pageTitle}{Summary: Deep Blue} % Assignment title
\newcommand{\hmwkDueDate}{Monday,\ September\ 15,\ 2017} % Due date
\newcommand{\courseName}{Artificial Intelligence Nanodegree} % Course/class
\newcommand{\authorName}{Johannes Stricker} % Your name

%----------------------------------------------------------------------------------------
%	TITLE PAGE
%----------------------------------------------------------------------------------------

\title{
\vspace{2in}
\textmd{\textbf{\courseName:\ \pageTitle}}\\
\normalsize\vspace{0.1in}\small{Due\ on\ \hmwkDueDate}\\
\vspace{0.1in}\large{\textit{\hmwkClassInstructor\ \hmwkClassTime}}
\vspace{3in}
}

\author{\textbf{\authorName}}
\date{} % Insert date here if you want it to appear below your name

%----------------------------------------------------------------------------------------

\begin{document}

  Deep Blue is a chess machine that was developed at IBM Research and defeated the
  chess grandmaster Garry Kasparov under tournament conditions in 1997. It is
  sometimes referred to as 'Deep Blue II', because it is based on an earlier version,
  the 'Deep Blue I', which had lost to grandmaster Kasparov a year earlier.

  Deep Blue II is designed as a massively parallel system with 30 processors, each
  featuring 16 chess chips, that can search about 2 million chess positions per second.
  The processors communicate with each other using the Message Passing Interface
  (short MPI) standard.

  The machine has a 3-layer architecture. One of the processors works as a master
  that's responsible for searching the top levels of the game tree. It then
  delegates further branch exploration to it's sibling processors. These search a
  few levels deeper, before they pass work on to the chess chips, that explore the
  last few levels of the tree. The chess chips are no general purpose processing
  units and therefore can only act as slaves. They consist of three parts: the
  move generator, the evaluation function and the search control. The option to
  extend the chess chips through external FPGAs exists, but was never used in practice.

  The move generator is implemented as an 8x8 array of combinatorical logic, which
  is essentially a representation of a chess board. It has a finite state machine
  that controls which moves are generated. The chip can compute all possible moves
  at a time and therefore operates with minimum latency. To improve performance it
  is preferable to search the moves in a best-first order, which the chip tries to estimate.

  The evaluation function is implemented in Hardware, which on the one hand increases
  it's speed, but on the other hand decreases the flexibility to adjust features.
  It is split into a fast and a slow evaluation, to be able to skip an expensive
  computation when an approximation is good enough. Essentially it is a sum of feature
  values, that range from very simple to very complex. A feature can be static or
  dynamic, meaning that it is scaled during the search. The evaluation function can
  be set up with a total of 8150 parameters and can detect about 8000 features,
  most of which were designed by hand.

  For the search algorithm a hybrid solution of hardware search for the lower branches
  and software search for the higher branches is used. The hardware search is
  implemented directly on the chess chip. It is a fixed depth null-window search,
  which includes a quiescence search. To achieve a good balance between the more
  performant hardware search and the more complex software search, the hardware search
  typically only searches 4 to 5 ply deep.
  During the hardware search, the host processor can do other work and has to poll
  the chip to determine when the search has been finished.

  The software search is supposed to be highly selective, meaning that branches
  that seem to have a high relative utility - among other characteristics - receive
  credit, which can then be 'cashed in' to examine the branch deeper. A problem
  with this occurs when a particular branch is the current best branch for both
  players, in which case both players would accumulate infinite credit and the search
  would explode. To counteract that, the so called 'dual credit' mechanism reduces
  one players credit when the other player cashes his credit in.
  The software search algorithm is implemented as a null-window alpha-beta search.
  Because the overall search is highly parallelized, it is non-deterministic. This
  makes debugging much more difficult, but apart from that is not an issue.

  Deep Blue features an opening book that was designed by the hands of four chess
  grandmasters. It consists of about 4000 positions.
  In addition to the standard opening book, Deep Blue can also fall back on a large
  grandmaster game database to create an extended opening book. The idea behind that
  is to assign values to moves based on a number of factors, e.g. the number of times
  a move has been played or the strength of the player playing it, and then play
  the move with the highest value. The extended opening book was used with good
  success in the matches against Kasparov.

  The chess chips also have access to an endgame database, which includes all chess
  positions with five or fewer pieces on the board, however these did not play a
  critical role in the matches against Kasparov.

  Because chess matches are usually played with a time limit, Deep Blue features
  a time control mechanism. Before each search, two time targets are set: if a
  reasonable move can be found within the first time limit, that move will be played.
  Under a few conditions however, Deep Blue will continue to search for a better
  move, until the second time limit is reached or the conditions can be met. This
  mechanism ensures that Deep Blue is able to finish a match in the given time,
  but also take enough time for harder decisions.

\end{document}
